%%%%%%%%%%%%%%%%%%%%%%%%%%%%%%%%%%%%%
% Document properties and packages
%%%%%%%%%%%%%%%%%%%%%%%%%%%%%%%%%%%%%
\documentclass[a4paper,10pt,final]{book}

% misc
\renewcommand{\familydefault}{bch}	% font
\pagestyle{plain}					% no pagenumbering
\setlength{\parindent}{0pt}			% no paragraph indentation

% required packages (add your own)
\usepackage{graphicx}
\usepackage[top=2cm,left=2cm,right=2cm,bottom=2cm]{geometry} % margins
\usepackage{epic}
\usepackage{figsize}
\usepackage{float}
\usepackage[pdfpagelabels=false,bookmarks]{hyperref}
\usepackage[section]{placeins}
\usepackage{cite}
\usepackage{epsfig}
\usepackage{amsmath}
\usepackage{amssymb}
\usepackage{amsfonts}
\usepackage{url}
\usepackage{listings}
\usepackage[utf8]{inputenc}
\usepackage{csquotes}
\usepackage[usenames,dvipsnames]{xcolor} % color
\usepackage{tikz}
\usepackage{multicol}										% columns env.
	\setlength{\columnsep}{1cm}
\usepackage{paralist}										% compact lists


%%%%%%%%%%%%%%%%%%%%%%%%%%%%%%%%%%%%%
% define macros (for convience)
%%%%%%%%%%%%%%%%%%%%%%%%%%%%%%%%%%%%%
\newcommand{\Sep}{\vspace{1.5mm}}
\newcommand{\SmallSep}{\vspace{0.5mm}}
\newcommand{\DocTitle}{Star Wars Cypher}
\newcommand{\DocSubtitle}{Playing in A Galaxy Far Far Away using the Cypher System}
\newcommand{\DocAuthors}{JC Bailey \& Greg Linklater}
\newcommand{\DocKeywords}{}
\newcommand{\DocColor}{NavyBlue}

\definecolor{mygreen}{rgb}{0,0.6,0}
\definecolor{mygray}{rgb}{0.5,0.5,0.5}
\definecolor{mymauve}{rgb}{0.58,0,0.82}
\definecolor{webgreen}{rgb}{0,.5,0}
\definecolor{webbrown}{rgb}{.6,0,0}
\definecolor{RoyalBlue}{rgb}{0.0, 0.14, 0.4}

%%%%%%%%%%%%%%%%%%%%%%%%%%%%%%%%%%%%%
% My special commands
%%%%%%%%%%%%%%%%%%%%%%%%%%%%%%%%%%%%%
\newcommand{\HRule}{\rule{\linewidth}{0.5mm}} % Defines a new command for the horizontal lines, change thickness here
\newcommand{\newSection}[1]{\section*{#1} \addcontentsline{toc}{section}{#1} \label{sec:#1} \HRule}

%%%%%%%%%%%%%%%%%%%%%%%%%%%%%%%%%%%%%
% Section Environment
%%%%%%%%%%%%%%%%%%%%%%%%%%%%%%%%%%%%%
\newenvironment{docsection}[1]
{
  \begin{multicols*}{2}[\newSection{#1}]
}
{
  \end{multicols*}
  \newpage
}

\newenvironment{docBlankPage}
{
	\begin{center}
	\vspace*{\fill}
}
{
	\vspace*{\fill}
	\end{center}
}

%%%%%%%%%%%%%%%%%%%%%%%%%%%%%%%%%%%%%
% Pre-document setup
%%%%%%%%%%%%%%%%%%%%%%%%%%%%%%%%%%%%%

\usetikzlibrary{arrows,shapes}
\hypersetup{
%draft, % Uncomment to remove all links (useful for printing in black and white)
colorlinks=true, breaklinks=true, %bookmarks=true,bookmarksnumbered,
urlcolor=webbrown, linkcolor=RoyalBlue, citecolor=webgreen, % Link colors
filecolor=\DocColor,
pdftitle={\DocTitle}, % PDF title
pdfauthor={\DocAuthors}, % PDF Author
pdfsubject={\DocSubtitle}, % PDF Subject
pdfkeywords={\DocKeywords}, % PDF Keywords
pdfcreator={pdfLaTeX}, % PDF Creator
pdfproducer={LaTeX}, % PDF producer
}

%%%%%%%%%%%%%%%%%%%%%%%%%%%%%%%%%%%%%
% Begin document
%%%%%%%%%%%%%%%%%%%%%%%%%%%%%%%%%%%%%
\begin{document}
\frontmatter

%%%%%%%%%%%%%%%%%%%%%%%%%%%%%%%%%%%%%
% Title Page
%%%%%%%%%%%%%%%%%%%%%%%%%%%%%%%%%%%%%
\begin{titlepage}
\center
\vspace*{\fill}
\HRule \\[4mm]

% Doc name
{ \huge \textsc{\DocTitle} }\\ % Title of your document
{ \small \textit{\DocSubtitle} }\\[3mm]

% Author
\emph{Written by:}\\
{\small \textsc{\DocAuthors}} \\% Your name

\HRule \\[3mm]
\vspace*{\fill}
\end{titlepage}


\pagestyle{plain}
\begin{docBlankPage}
\textit{This page is left intentionally blank.}
\end{docBlankPage}
%%%%%%%%%%%%%%%%%%%%%%%%%%%%%%%%%%%%%
% TOC
%%%%%%%%%%%%%%%%%%%%%%%%%%%%%%%%%%%%%
\tableofcontents
\mainmatter % This stupid fucking function adds a blank page and I dont know why.

%%%%%%%%%%%%%%%%%%%%%%%%%%%%%%%%%%%%%
% Begin content
%%%%%%%%%%%%%%%%%%%%%%%%%%%%%%%%%%%%%

% Intro
\begin{docsection}{How To Play}

The rules of the Cypher System are quite straightforward at their heart, as 
all of gameplay is based around a few core concepts. This chapter provides a
 brief explanation of how to play the game, and it’s useful for learning the 
 game. \\

The Cypher System uses a twenty-sided die (1d20) to determine the results of most actions. 
Whenever a roll of any kind is called for and no die is specified, roll a d20.
The game master sets a difficulty for any given task. There are ten degrees of difficulty. 
Thus, the difficulty of a task can be rated on a scale of 1 to 10. \\

Each difficulty has a target number
associated with it. The target number is
always three times the task’s difficulty, so
a difficulty 4 task has a target number of
12. To succeed at the task, you must roll
the target number or higher. \\

Character skills, favorable circumstances,
or excellent equipment can decrease
the difficulty of a task. \\

For example, if a character is trained in climbing, she turns
a difficulty 6 climb into a difficulty 5 climb.
This is called decreasing the difficulty by
one step. If she is specialized in climbing,
she turns a difficulty 6 climb into a difficulty
4 climb. This is called decreasing the
difficulty by two steps. \\


\end{docsection}

%Front matter
\begin{docsection}{Some Other Stuff}

Space to be filled with content. This should also be a multi-column page.

\end{docsection}

%Descriptors
\begin{docsection}{Descriptors}

Space to be filled with content. This should also be a multi-column page.

\end{docsection}

%Types
\begin{docsection}{Types}

Players may choose from the following types for their characters: \hyperlink{sub:soldier}{Soldier}, \hyperlink{sub:scout}{Scout}, \hyperlink{sub:scoundrel_noble}{Scoundrel or Noble}.

\subsection*{Soldier} % (fold)
\label{sub:soldier}

The general warrior.

% subsection* soldier (end)

\subsection*{Scout} % (fold)
\label{sub:scout}

The fast, typically stealthy operative.

% subsection* scout (end)

\subsection*{Scoundrel / Noble} % (fold)
\label{sub:scoundrel_noble}

The silver-tongued smuggler or diplomat.

% subsection* scoundrel_noble (end)

\end{docsection}

%Foci
\begin{docsection}{Foci}

Space to be filled with content. This should also be a multi-column page.

\end{docsection}

%The Force
\begin{docsection}{The Force}

Unlike the Star Wars Saga rule set and other homebrew systems, these rules do not restrict force sensitivity to a particular class; force sensitivity is instead treated as an additional skill tree that can be pursued by players at the DM's discretion.

\end{docsection}

%%%%%%%%%%%%%%%%%%%%%%%%%%%%%%%%%%%%%
% End document
%%%%%%%%%%%%%%%%%%%%%%%%%%%%%%%%%%%%%
\end{document}
